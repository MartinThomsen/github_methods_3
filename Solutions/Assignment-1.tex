% Options for packages loaded elsewhere
\PassOptionsToPackage{unicode}{hyperref}
\PassOptionsToPackage{hyphens}{url}
%
\documentclass[
]{article}
\usepackage{amsmath,amssymb}
\usepackage{lmodern}
\usepackage{ifxetex,ifluatex}
\ifnum 0\ifxetex 1\fi\ifluatex 1\fi=0 % if pdftex
  \usepackage[T1]{fontenc}
  \usepackage[utf8]{inputenc}
  \usepackage{textcomp} % provide euro and other symbols
\else % if luatex or xetex
  \usepackage{unicode-math}
  \defaultfontfeatures{Scale=MatchLowercase}
  \defaultfontfeatures[\rmfamily]{Ligatures=TeX,Scale=1}
\fi
% Use upquote if available, for straight quotes in verbatim environments
\IfFileExists{upquote.sty}{\usepackage{upquote}}{}
\IfFileExists{microtype.sty}{% use microtype if available
  \usepackage[]{microtype}
  \UseMicrotypeSet[protrusion]{basicmath} % disable protrusion for tt fonts
}{}
\makeatletter
\@ifundefined{KOMAClassName}{% if non-KOMA class
  \IfFileExists{parskip.sty}{%
    \usepackage{parskip}
  }{% else
    \setlength{\parindent}{0pt}
    \setlength{\parskip}{6pt plus 2pt minus 1pt}}
}{% if KOMA class
  \KOMAoptions{parskip=half}}
\makeatother
\usepackage{xcolor}
\IfFileExists{xurl.sty}{\usepackage{xurl}}{} % add URL line breaks if available
\IfFileExists{bookmark.sty}{\usepackage{bookmark}}{\usepackage{hyperref}}
\hypersetup{
  pdftitle={practical\_exercise\_2, Methods 3, 2021, autumn semester},
  pdfauthor={{[}FILL IN YOUR NAME{]}},
  hidelinks,
  pdfcreator={LaTeX via pandoc}}
\urlstyle{same} % disable monospaced font for URLs
\usepackage[margin=1in]{geometry}
\usepackage{color}
\usepackage{fancyvrb}
\newcommand{\VerbBar}{|}
\newcommand{\VERB}{\Verb[commandchars=\\\{\}]}
\DefineVerbatimEnvironment{Highlighting}{Verbatim}{commandchars=\\\{\}}
% Add ',fontsize=\small' for more characters per line
\usepackage{framed}
\definecolor{shadecolor}{RGB}{248,248,248}
\newenvironment{Shaded}{\begin{snugshade}}{\end{snugshade}}
\newcommand{\AlertTok}[1]{\textcolor[rgb]{0.94,0.16,0.16}{#1}}
\newcommand{\AnnotationTok}[1]{\textcolor[rgb]{0.56,0.35,0.01}{\textbf{\textit{#1}}}}
\newcommand{\AttributeTok}[1]{\textcolor[rgb]{0.77,0.63,0.00}{#1}}
\newcommand{\BaseNTok}[1]{\textcolor[rgb]{0.00,0.00,0.81}{#1}}
\newcommand{\BuiltInTok}[1]{#1}
\newcommand{\CharTok}[1]{\textcolor[rgb]{0.31,0.60,0.02}{#1}}
\newcommand{\CommentTok}[1]{\textcolor[rgb]{0.56,0.35,0.01}{\textit{#1}}}
\newcommand{\CommentVarTok}[1]{\textcolor[rgb]{0.56,0.35,0.01}{\textbf{\textit{#1}}}}
\newcommand{\ConstantTok}[1]{\textcolor[rgb]{0.00,0.00,0.00}{#1}}
\newcommand{\ControlFlowTok}[1]{\textcolor[rgb]{0.13,0.29,0.53}{\textbf{#1}}}
\newcommand{\DataTypeTok}[1]{\textcolor[rgb]{0.13,0.29,0.53}{#1}}
\newcommand{\DecValTok}[1]{\textcolor[rgb]{0.00,0.00,0.81}{#1}}
\newcommand{\DocumentationTok}[1]{\textcolor[rgb]{0.56,0.35,0.01}{\textbf{\textit{#1}}}}
\newcommand{\ErrorTok}[1]{\textcolor[rgb]{0.64,0.00,0.00}{\textbf{#1}}}
\newcommand{\ExtensionTok}[1]{#1}
\newcommand{\FloatTok}[1]{\textcolor[rgb]{0.00,0.00,0.81}{#1}}
\newcommand{\FunctionTok}[1]{\textcolor[rgb]{0.00,0.00,0.00}{#1}}
\newcommand{\ImportTok}[1]{#1}
\newcommand{\InformationTok}[1]{\textcolor[rgb]{0.56,0.35,0.01}{\textbf{\textit{#1}}}}
\newcommand{\KeywordTok}[1]{\textcolor[rgb]{0.13,0.29,0.53}{\textbf{#1}}}
\newcommand{\NormalTok}[1]{#1}
\newcommand{\OperatorTok}[1]{\textcolor[rgb]{0.81,0.36,0.00}{\textbf{#1}}}
\newcommand{\OtherTok}[1]{\textcolor[rgb]{0.56,0.35,0.01}{#1}}
\newcommand{\PreprocessorTok}[1]{\textcolor[rgb]{0.56,0.35,0.01}{\textit{#1}}}
\newcommand{\RegionMarkerTok}[1]{#1}
\newcommand{\SpecialCharTok}[1]{\textcolor[rgb]{0.00,0.00,0.00}{#1}}
\newcommand{\SpecialStringTok}[1]{\textcolor[rgb]{0.31,0.60,0.02}{#1}}
\newcommand{\StringTok}[1]{\textcolor[rgb]{0.31,0.60,0.02}{#1}}
\newcommand{\VariableTok}[1]{\textcolor[rgb]{0.00,0.00,0.00}{#1}}
\newcommand{\VerbatimStringTok}[1]{\textcolor[rgb]{0.31,0.60,0.02}{#1}}
\newcommand{\WarningTok}[1]{\textcolor[rgb]{0.56,0.35,0.01}{\textbf{\textit{#1}}}}
\usepackage{graphicx}
\makeatletter
\def\maxwidth{\ifdim\Gin@nat@width>\linewidth\linewidth\else\Gin@nat@width\fi}
\def\maxheight{\ifdim\Gin@nat@height>\textheight\textheight\else\Gin@nat@height\fi}
\makeatother
% Scale images if necessary, so that they will not overflow the page
% margins by default, and it is still possible to overwrite the defaults
% using explicit options in \includegraphics[width, height, ...]{}
\setkeys{Gin}{width=\maxwidth,height=\maxheight,keepaspectratio}
% Set default figure placement to htbp
\makeatletter
\def\fps@figure{htbp}
\makeatother
\setlength{\emergencystretch}{3em} % prevent overfull lines
\providecommand{\tightlist}{%
  \setlength{\itemsep}{0pt}\setlength{\parskip}{0pt}}
\setcounter{secnumdepth}{-\maxdimen} % remove section numbering
\ifluatex
  \usepackage{selnolig}  % disable illegal ligatures
\fi

\title{practical\_exercise\_2, Methods 3, 2021, autumn semester}
\author{{[}FILL IN YOUR NAME{]}}
\date{{[}FILL IN THE DATE{]}}

\begin{document}
\maketitle

\hypertarget{assignment-1-using-mixed-effects-modelling-to-model-hierarchical-data}{%
\section{Assignment 1: Using mixed effects modelling to model
hierarchical
data}\label{assignment-1-using-mixed-effects-modelling-to-model-hierarchical-data}}

In this assignment we will be investigating the \emph{politeness}
dataset of Winter and Grawunder (2012) and apply basic methods of
multilevel modelling.

\hypertarget{dataset}{%
\subsection{Dataset}\label{dataset}}

The dataset has been shared on GitHub, so make sure that the csv-file is
on your current path. Otherwise you can supply the full path.

\begin{Shaded}
\begin{Highlighting}[]
\NormalTok{politeness }\OtherTok{\textless{}{-}} \FunctionTok{read.csv}\NormalTok{(}\StringTok{\textquotesingle{}\textasciitilde{}/Uni/7 semester/Multilevel statistical modeling and machine learning/github\_methods\_3/week\_02/politeness.csv\textquotesingle{}}\NormalTok{) }\DocumentationTok{\#\# read in data}
\end{Highlighting}
\end{Shaded}

\hypertarget{exercises-and-objectives}{%
\section{Exercises and objectives}\label{exercises-and-objectives}}

The objectives of the exercises of this assignment are:\\
1) Learning to recognize hierarchical structures within datasets and
describing them\\
2) Creating simple multilevel models and assessing their fitness\\
3) Write up a report about the findings of the study

REMEMBER: In your report, make sure to include code that can reproduce
the answers requested in the exercises below\\
REMEMBER: This assignment will be part of your final portfolio

\hypertarget{exercise-1---describing-the-dataset-and-making-some-initial-plots}{%
\subsection{Exercise 1 - describing the dataset and making some initial
plots}\label{exercise-1---describing-the-dataset-and-making-some-initial-plots}}

\begin{enumerate}
\def\labelenumi{\arabic{enumi})}
\tightlist
\item
  Describe the dataset, such that someone who happened upon this dataset
  could understand the variables and what they contain

  \begin{enumerate}
  \def\labelenumii{\roman{enumii}.}
  \tightlist
  \item
    Also consider whether any of the variables in \emph{politeness}
    should be encoded as factors or have the factor encoding removed.
    Hint: \texttt{?factor}\\
  \end{enumerate}
\item
  Create a new data frame that just contains the subject \emph{F1} and
  run two linear models; one that expresses \emph{f0mn} as dependent on
  \emph{scenario} as an integer; and one that expresses \emph{f0mn} as
  dependent on \emph{scenario} encoded as a factor

  \begin{enumerate}
  \def\labelenumii{\roman{enumii}.}
  \tightlist
  \item
    Include the model matrices, \(X\) from the General Linear Model, for
    these two models in your report and describe the different
    interpretations of \emph{scenario} that these entail
  \item
    Which coding of \emph{scenario}, as a factor or not, is more
    fitting?
  \end{enumerate}
\item
  Make a plot that includes a subplot for each subject that has
  \emph{scenario} on the x-axis and \emph{f0mn} on the y-axis and where
  points are colour coded according to \emph{attitude}

  \begin{enumerate}
  \def\labelenumii{\roman{enumii}.}
  \tightlist
  \item
    Describe the differences between subjects
  \end{enumerate}
\end{enumerate}

\hypertarget{exercise-2---comparison-of-models}{%
\subsection{Exercise 2 - comparison of
models}\label{exercise-2---comparison-of-models}}

For this part, make sure to have \texttt{lme4} installed.\\
You can install it using \texttt{install.packages("lme4")} and load it
using \texttt{library(lme4)}\\
\texttt{lmer} is used for multilevel modelling

\begin{Shaded}
\begin{Highlighting}[]
\NormalTok{mixed.model }\OtherTok{\textless{}{-}} \FunctionTok{lmer}\NormalTok{(}\AttributeTok{formula=}\NormalTok{..., }\AttributeTok{data=}\NormalTok{...)}
\NormalTok{example.formula }\OtherTok{\textless{}{-}} \FunctionTok{formula}\NormalTok{(dep.variable }\SpecialCharTok{\textasciitilde{}}\NormalTok{ first.level.variable }\SpecialCharTok{+}\NormalTok{ (}\DecValTok{1} \SpecialCharTok{|}\NormalTok{ second.level.variable))}
\end{Highlighting}
\end{Shaded}

\begin{enumerate}
\def\labelenumi{\arabic{enumi})}
\tightlist
\item
  Build four models and do some comparisons

  \begin{enumerate}
  \def\labelenumii{\roman{enumii}.}
  \tightlist
  \item
    a single level model that models \emph{f0mn} as dependent on
    \emph{gender}
  \item
    a two-level model that adds a second level on top of i. where unique
    intercepts are modelled for each \emph{scenario}
  \item
    a two-level model that only has \emph{subject} as an intercept
  \item
    a two-level model that models intercepts for both \emph{scenario}
    and \emph{subject}
  \item
    which of the models has the lowest residual standard deviation, also
    compare the Akaike Information Criterion \texttt{AIC}?
  \item
    which of the second-level effects explains the most variance?
  \end{enumerate}
\item
  Why is our single-level model bad?

  \begin{enumerate}
  \def\labelenumii{\roman{enumii}.}
  \tightlist
  \item
    create a new data frame that has three variables, \emph{subject},
    \emph{gender} and \emph{f0mn}, where \emph{f0mn} is the average of
    all responses of each subject, i.e.~averaging across \emph{attitude}
    and\_scenario\_
  \item
    build a single-level model that models \emph{f0mn} as dependent on
    \emph{gender} using this new dataset
  \item
    make Quantile-Quantile plots, comparing theoretical quantiles to the
    sample quantiles) using \texttt{qqnorm} and \texttt{qqline} for the
    new single-level model and compare it to the old single-level model
    (from 1).i). Which model's residuals (\(\epsilon\)) fulfil the
    assumptions of the General Linear Model better?)
  \item
    Also make a quantile-quantile plot for the residuals of the
    multilevel model with two intercepts. Does it look alright?
  \end{enumerate}
\item
  Plotting the two-intercepts model

  \begin{enumerate}
  \def\labelenumii{\roman{enumii}.}
  \tightlist
  \item
    Create a plot for each subject, (similar to part 3 in Exercise 1),
    this time also indicating the fitted value for each of the subjects
    for each for the scenarios (hint use \texttt{fixef} to get the
    ``grand effects'' for each gender and \texttt{ranef} to get the
    subject- and scenario-specific effects)
  \end{enumerate}
\end{enumerate}

\hypertarget{exercise-3---now-with-attitude}{%
\subsection{Exercise 3 - now with
attitude}\label{exercise-3---now-with-attitude}}

\begin{enumerate}
\def\labelenumi{\arabic{enumi})}
\tightlist
\item
  Carry on with the model with the two unique intercepts fitted
  (\emph{scenario} and \emph{subject}).

  \begin{enumerate}
  \def\labelenumii{\roman{enumii}.}
  \tightlist
  \item
    now build a model that has \emph{attitude} as a main effect besides
    \emph{gender}
  \item
    make a separate model that besides the main effects of
    \emph{attitude} and \emph{gender} also include their interaction
  \item
    describe what the interaction term in the model says about Korean
    men's pitch when they are polite relative to Korean women's pitch
    when they are polite (you don't have to judge whether it is
    interesting)\\
  \end{enumerate}
\item
  Compare the three models (1. gender as a main effect; 2. gender and
  attitude as main effects; 3. gender and attitude as main effects and
  the interaction between them. For all three models model unique
  intercepts for \emph{subject} and \emph{scenario}) using residual
  variance, residual standard deviation and AIC.\\
\item
  Choose the model that you think describe the data the best - and write
  a short report on the main findings based on this model. At least
  include the following:
\end{enumerate}

\begin{enumerate}
\def\labelenumi{\roman{enumi}.}
\tightlist
\item
  describe what the dataset consists of\\
\item
  what can you conclude about the effect of gender and attitude on pitch
  (if anything)?\\
\item
  motivate why you would include separate intercepts for subjects and
  scenarios (if you think they should be included)\\
\item
  describe the variance components of the second level (if any)\\
\item
  include a Quantile-Quantile plot of your chosen model
\end{enumerate}

\end{document}
